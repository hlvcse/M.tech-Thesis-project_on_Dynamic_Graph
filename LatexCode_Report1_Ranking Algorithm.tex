%anyone can edit with this link: https://www.overleaf.com/2911146382nfbrdxdcjvmx
%anyone can view with this link: https://www.overleaf.com/read/wvdtbnsczgbw




%% bare_conf_compsoc.tex
%% V1.4b
%% 2015/08/26
%% by Michael Shell
%% See:
%% http://www.michaelshell.org/
%% for current contact information.
%%
%% This is a skeleton file demonstrating the use of IEEEtran.cls
%% (requires IEEEtran.cls version 1.8b or later) with an IEEE Computer
%% Society conference paper.
%%
%% Support sites:
%% http://www.michaelshell.org/tex/ieeetran/
%% http://www.ctan.org/pkg/ieeetran
%% and
%% http://www.ieee.org/

%%*************************************************************************
%% Legal Notice:
%% This code is offered as-is without any warranty either expressed or
%% implied; without even the implied warranty of MERCHANTABILITY or
%% FITNESS FOR A PARTICULAR PURPOSE! 
%% User assumes all risk.
%% In no event shall the IEEE or any contributor to this code be liable for
%% any damages or losses, including, but not limited to, incidental,
%% consequential, or any other damages, resulting from the use or misuse
%% of any information contained here.
%%
%% All comments are the opinions of their respective authors and are not
%% necessarily endorsed by the IEEE.
%%
%% This work is distributed under the LaTeX Project Public License (LPPL)
%% ( http://www.latex-project.org/ ) version 1.3, and may be freely used,
%% distributed and modified. A copy of the LPPL, version 1.3, is included
%% in the base LaTeX documentation of all distributions of LaTeX released
%% 2003/12/01 or later.
%% Retain all contribution notices and credits.
%% ** Modified files should be clearly indicated as such, including  **
%% ** renaming them and changing author support contact information. **
%%*************************************************************************


% *** Authors should verify (and, if needed, correct) their LaTeX system  ***
% *** with the testflow diagnostic prior to trusting their LaTeX platform ***
% *** with production work. The IEEE's font choices and paper sizes can   ***
% *** trigger bugs that do not appear when using other class files.       ***                          ***
% The testflow support page is at:
% http://www.michaelshell.org/tex/testflow/



\documentclass[conference,compsoc]{IEEEtran}
% Some/most Computer Society conferences require the compsoc mode option,
% but others may want the standard conference format.
%
% If IEEEtran.cls has not been installed into the LaTeX system files,
% manually specify the path to it like:
% \documentclass[conference,compsoc]{../sty/IEEEtran}





% Some very useful LaTeX packages include:
% (uncomment the ones you want to load)


% *** MISC UTILITY PACKAGES ***
%
%\usepackage{ifpdf}
% Heiko Oberdiek's ifpdf.sty is very useful if you need conditional
% compilation based on whether the output is pdf or dvi.
% usage:
% \ifpdf
%   % pdf code
% \else
%   % dvi code
% \fi
% The latest version of ifpdf.sty can be obtained from:
% http://www.ctan.org/pkg/ifpdf
% Also, note that IEEEtran.cls V1.7 and later provides a builtin
% \ifCLASSINFOpdf conditional that works the same way.
% When switching from latex to pdflatex and vice-versa, the compiler may
% have to be run twice to clear warning/error messages.






% *** CITATION PACKAGES ***
%
\documentclass{article}
\usepackage{graphicx}
\graphicspath{ {./images/} }

\ifCLASSOPTIONcompsoc
  % IEEE Computer Society needs nocompress option
  % requires cite.sty v4.0 or later (November 2003)
  \usepackage[nocompress]{cite}
\else
  % normal IEEE
  \usepackage{cite}
\fi
% cite.sty was written by Donald Arseneau
% V1.6 and later of IEEEtran pre-defines the format of the cite.sty package
% \cite{} output to follow that of the IEEE. Loading the cite package will
% result in citation numbers being automatically sorted and properly
% "compressed/ranged". e.g., [1], [9], [2], [7], [5], [6] without using
% cite.sty will become [1], [2], [5]--[7], [9] using cite.sty. cite.sty's
% \cite will automatically add leading space, if needed. Use cite.sty's
% noadjust option (cite.sty V3.8 and later) if you want to turn this off
% such as if a citation ever needs to be enclosed in parenthesis.
% cite.sty is already installed on most LaTeX systems. Be sure and use
% version 5.0 (2009-03-20) and later if using hyperref.sty.
% The latest version can be obtained at:
% http://www.ctan.org/pkg/cite
% The documentation is contained in the cite.sty file itself.
%
% Note that some packages require special options to format as the Computer
% Society requires. In particular, Computer Society  papers do not use
% compressed citation ranges as is done in typical IEEE papers
% (e.g., [1]-[4]). Instead, they list every citation separately in order
% (e.g., [1], [2], [3], [4]). To get the latter we need to load the cite
% package with the nocompress option which is supported by cite.sty v4.0
% and later.





% *** GRAPHICS RELATED PACKAGES ***
%
\ifCLASSINFOpdf
  % \usepackage[pdftex]{graphicx}
  % declare the path(s) where your graphic files are
  % \graphicspath{{../pdf/}{../jpeg/}}
  % and their extensions so you won't have to specify these with
  % every instance of \includegraphics
  % \DeclareGraphicsExtensions{.pdf,.jpeg,.png}
\else
  % or other class option (dvipsone, dvipdf, if not using dvips). graphicx
  % will default to the driver specified in the system graphics.cfg if no
  % driver is specified.
  % \usepackage[dvips]{graphicx}
  % declare the path(s) where your graphic files are
  % \graphicspath{{../eps/}}
  % and their extensions so you won't have to specify these with
  % every instance of \includegraphics
  % \DeclareGraphicsExtensions{.eps}
\fi
% graphicx was written by David Carlisle and Sebastian Rahtz. It is
% required if you want graphics, photos, etc. graphicx.sty is already
% installed on most LaTeX systems. The latest version and documentation
% can be obtained at: 
% http://www.ctan.org/pkg/graphicx
% Another good source of documentation is "Using Imported Graphics in
% LaTeX2e" by Keith Reckdahl which can be found at:
% http://www.ctan.org/pkg/epslatex
%
% latex, and pdflatex in dvi mode, support graphics in encapsulated
% postscript (.eps) format. pdflatex in pdf mode supports graphics
% in .pdf, .jpeg, .png and .mps (metapost) formats. Users should ensure
% that all non-photo figures use a vector format (.eps, .pdf, .mps) and
% not a bitmapped formats (.jpeg, .png). The IEEE frowns on bitmapped formats
% which can result in "jaggedy"/blurry rendering of lines and letters as
% well as large increases in file sizes.
%
% You can find documentation about the pdfTeX application at:
% http://www.tug.org/applications/pdftex





% *** MATH PACKAGES ***
%
%\usepackage{amsmath}
% A popular package from the American Mathematical Society that provides
% many useful and powerful commands for dealing with mathematics.
%
% Note that the amsmath package sets \interdisplaylinepenalty to 10000
% thus preventing page breaks from occurring within multiline equations. Use:
%\interdisplaylinepenalty=2500
% after loading amsmath to restore such page breaks as IEEEtran.cls normally
% does. amsmath.sty is already installed on most LaTeX systems. The latest
% version and documentation can be obtained at:
% http://www.ctan.org/pkg/amsmath





% *** SPECIALIZED LIST PACKAGES ***
%
%\usepackage{algorithmic}
% algorithmic.sty was written by Peter Williams and Rogerio Brito.
% This package provides an algorithmic environment fo describing algorithms.
% You can use the algorithmic environment in-text or within a figure
% environment to provide for a floating algorithm. Do NOT use the algorithm
% floating environment provided by algorithm.sty (by the same authors) or
% algorithm2e.sty (by Christophe Fiorio) as the IEEE does not use dedicated
% algorithm float types and packages that provide these will not provide
% correct IEEE style captions. The latest version and documentation of
% algorithmic.sty can be obtained at:
% http://www.ctan.org/pkg/algorithms
% Also of interest may be the (relatively newer and more customizable)
% algorithmicx.sty package by Szasz Janos:
% http://www.ctan.org/pkg/algorithmicx




% *** ALIGNMENT PACKAGES ***
%
%\usepackage{array}
% Frank Mittelbach's and David Carlisle's array.sty patches and improves
% the standard LaTeX2e array and tabular environments to provide better
% appearance and additional user controls. As the default LaTeX2e table
% generation code is lacking to the point of almost being broken with
% respect to the quality of the end results, all users are strongly
% advised to use an enhanced (at the very least that provided by array.sty)
% set of table tools. array.sty is already installed on most systems. The
% latest version and documentation can be obtained at:
% http://www.ctan.org/pkg/array


% IEEEtran contains the IEEEeqnarray family of commands that can be used to
% generate multiline equations as well as matrices, tables, etc., of high
% quality.




% *** SUBFIGURE PACKAGES ***
%\ifCLASSOPTIONcompsoc
%  \usepackage[caption=false,font=footnotesize,labelfont=sf,textfont=sf]{subfig}
%\else
%  \usepackage[caption=false,font=footnotesize]{subfig}
%\fi
% subfig.sty, written by Steven Douglas Cochran, is the modern replacement
% for subfigure.sty, the latter of which is no longer maintained and is
% incompatible with some LaTeX packages including fixltx2e. However,
% subfig.sty requires and automatically loads Axel Sommerfeldt's caption.sty
% which will override IEEEtran.cls' handling of captions and this will result
% in non-IEEE style figure/table captions. To prevent this problem, be sure
% and invoke subfig.sty's "caption=false" package option (available since
% subfig.sty version 1.3, 2005/06/28) as this is will preserve IEEEtran.cls
% handling of captions.
% Note that the Computer Society format requires a sans serif font rather
% than the serif font used in traditional IEEE formatting and thus the need
% to invoke different subfig.sty package options depending on whether
% compsoc mode has been enabled.
%
% The latest version and documentation of subfig.sty can be obtained at:
% http://www.ctan.org/pkg/subfig




% *** FLOAT PACKAGES ***
%
%\usepackage{fixltx2e}
% fixltx2e, the successor to the earlier fix2col.sty, was written by
% Frank Mittelbach and David Carlisle. This package corrects a few problems
% in the LaTeX2e kernel, the most notable of which is that in current
% LaTeX2e releases, the ordering of single and double column floats is not
% guaranteed to be preserved. Thus, an unpatched LaTeX2e can allow a
% single column figure to be placed prior to an earlier double column
% figure.
% Be aware that LaTeX2e kernels dated 2015 and later have fixltx2e.sty's
% corrections already built into the system in which case a warning will
% be issued if an attempt is made to load fixltx2e.sty as it is no longer
% needed.
% The latest version and documentation can be found at:
% http://www.ctan.org/pkg/fixltx2e


%\usepackage{stfloats}
% stfloats.sty was written by Sigitas Tolusis. This package gives LaTeX2e
% the ability to do double column floats at the bottom of the page as well
% as the top. (e.g., "\begin{figure*}[!b]" is not normally possible in
% LaTeX2e). It also provides a command:
%\fnbelowfloat
% to enable the placement of footnotes below bottom floats (the standard
% LaTeX2e kernel puts them above bottom floats). This is an invasive package
% which rewrites many portions of the LaTeX2e float routines. It may not work
% with other packages that modify the LaTeX2e float routines. The latest
% version and documentation can be obtained at:
% http://www.ctan.org/pkg/stfloats
% Do not use the stfloats baselinefloat ability as the IEEE does not allow
% \baselineskip to stretch. Authors submitting work to the IEEE should note
% that the IEEE rarely uses double column equations and that authors should try
% to avoid such use. Do not be tempted to use the cuted.sty or midfloat.sty
% packages (also by Sigitas Tolusis) as the IEEE does not format its papers in
% such ways.
% Do not attempt to use stfloats with fixltx2e as they are incompatible.
% Instead, use Morten Hogholm'a dblfloatfix which combines the features
% of both fixltx2e and stfloats:
%
% \usepackage{dblfloatfix}
% The latest version can be found at:
% http://www.ctan.org/pkg/dblfloatfix




% *** PDF, URL AND HYPERLINK PACKAGES ***
%
%\usepackage{url}
% url.sty was written by Donald Arseneau. It provides better support for
% handling and breaking URLs. url.sty is already installed on most LaTeX
% systems. The latest version and documentation can be obtained at:
% http://www.ctan.org/pkg/url
% Basically, \url{my_url_here}.




% *** Do not adjust lengths that control margins, column widths, etc. ***
% *** Do not use packages that alter fonts (such as pslatex).         ***
% There should be no need to do such things with IEEEtran.cls V1.6 and later.
% (Unless specifically asked to do so by the journal or conference you plan
% to submit to, of course. )


% correct bad hyphenation here
\hyphenation{op-tical net-works semi-conduc-tor}


\begin{document}
%
% paper title
% Titles are generally capitalized except for words such as a, an, and, as,
% at, but, by, for, in, nor, of, on, or, the, to and up, which are usually
% not capitalized unless they are the first or last word of the title.
% Linebreaks \\ can be used within to get better formatting as desired.
% Do not put math or special symbols in the title.
\title{Ranking Algorithm : Online Bipartite Matching}


% author names and affiliations
% use a multiple column layout for up to three different
% affiliations
\author{\IEEEauthorblockN{KM Pooja(2021CSM1011), Hanumat Lal Vishwakarma(2021CSM1019), Dr. Anil Shukla}\\
%\IEEEauthorblockN{Hanumat Lal Vishwakarma}\\
%\IEEEauthorblockN{Dr. Anil Shukla}\\
\IEEEauthorblockA{CSE Department, Indian Institute of Technology(IIT), Ropar (Academic Year 2021-23)\\
}
}

% conference papers do not typically use \thanks and this command
% is locked out in conference mode. If really needed, such as for
% the acknowledgment of grants, issue a \IEEEoverridecommandlockouts
% after \documentclass

% for over three affiliations, or if they all won't fit within the width
% of the page (and note that there is less available width in this regard for
% compsoc conferences compared to traditional conferences), use this
% alternative format:
% 
%\author{\IEEEauthorblockN{Michael Shell\IEEEauthorrefmark{1},
%Homer Simpson\IEEEauthorrefmark{2},
%James Kirk\IEEEauthorrefmark{3}, 
%Montgomery Scott\IEEEauthorrefmark{3} and
%Eldon Tyrell\IEEEauthorrefmark{4}}
%\IEEEauthorblockA{\IEEEauthorrefmark{1}School of Electrical and Computer Engineering\\
%Georgia Institute of Technology,
%Atlanta, Georgia 30332--0250\\ Email: see http://www.michaelshell.org/contact.html}
%\IEEEauthorblockA{\IEEEauthorrefmark{2}Twentieth Century Fox, Springfield, USA\\
%Email: homer@thesimpsons.com}
%\IEEEauthorblockA{\IEEEauthorrefmark{3}Starfleet Academy, San Francisco, California 96678-2391\\
%Telephone: (800) 555--1212, Fax: (888) 555--1212}
%\IEEEauthorblockA{\IEEEauthorrefmark{4}Tyrell Inc., 123 Replicant Street, Los Angeles, California 90210--4321}}




% use for special paper notices
%\IEEEspecialpapernotice{(Invited Paper)}




% make the title area
\maketitle

% As a general rule, do not put math, special symbols or citations
% in the abstract
%\begin{abstract}
%There has been a great deal of interest recently in the relative power of %on-line and off-line algorithms. An on-line algorithm receives a sequence of %requests and must respond to each request as soon as it is receiveD. An %off-line algorithm may wait until all requests have been received before %determining its responses. One approach to evaluating an on-line algorithm is %to compare its performance with that of the best possible off-line algorithm %for the same problem. However, a slightly different randomized algorithm, named %Ranking, due to Karp[?] performs much better. Thus, given a measure of %"profit", the performance of an on-line algorithm can be measured by the %worst-case ratio of its profit to that of the optimal off-line algorithm. 
%\end{abstract}

% no keywords




% For peer review papers, you can put extra information on the cover
% page as needed:
% \ifCLASSOPTIONpeerreview
% \begin{center} \bfseries EDICS Category: 3-BBND \end{center}
% \fi
%
% For peerreview papers, this IEEEtran command inserts a page break and
% creates the second title. It will be ignored for other modes.
\IEEEpeerreviewmaketitle



\section{Introduction}
% no \IEEEPARstart
In any typical algorithms, generally all the input data known beforehand. The algorithm can access the data in any order as the requirement. At the last, the algorithm produces the output. Such an algorithm is called \textbf{off-line algorithm}. \par
However, there are times when we can not see all the data before our algorithm must make some decisions.Stream mining, where we could store only a limited amount of the stream, and had to answer queries about the entire stream when called upon to do so. There is an extreme form of stream processing, where we must respond with an output after each stream element arrives. We thus must decide about each stream element knowing nothing at all of the future. Algorithms of this class are called \textbf{on-line algorithms}. Hence, an online algorithm must generate output without the knowledge of the entire input. Some common examples of online problems are:

\par
\begin{itemize}
  \item \noindent {\textbf{Scheduling:} {\normalfont{A set of jobs arriving sequentially must be scheduled on a set of machines, so as to minimize the completion time of the entire set of jobs. Jobs have to be scheduled on machines as they arrive
without knowing anything about future jobs.
}}}
\end{itemize}
 
 \par
 \begin{itemize}
  \item \noindent {\textbf{Routing:} {\normalfont Routing is the process of choosing paths on which data has to be sent in a communication network. Paths have to be selected as data arrives without the knowledge of future communication patterns.}}
\end{itemize}
 
 \par
 \begin{itemize}
  \item \noindent{\textbf{Paging:} {\normalfont The problem is to decide which pages to store in the cache memory that are likely to be referenced later, despite not knowing future page requests.
}}
\end{itemize}


Since online algorithms are forced to make local irrevocable choices, the output generated may not be optimal compared to the offline algorithm that has full knowledge about the input. Our interest lies in finding how good the online algorithm performs compared to the offline one. \par
\textbf{Online Bipartite matching} problem is also one such on-line algorithm where, There is a two set of vertices U and V. Set U is known but vertices $ v \in V $ arrives in online fashion. Edges are revealed once a vertices $ v \in V $ arrived. Algorithm need to match arrived vertices v to one of the available vertices in U.The main objective is to design an online algorithms with competitive ratios as close to 1 as possible(close to maximum size matching of off-line algorithm).
\textbf{Karp et al.} show that the competitiveness of the problem is trivial in the deterministic case. Any algorithm that always matches a vertex in U if a match is possible constructs a maximal matching, and therefore such an algorithm has a competitive ratio of 1/2. On the other hand, given any deterministic algorithm, it is easy to construct an input that forces that algorithm to find a matching of size no greater than half of the optimum.\par
The competitiveness of the problem in the randomized case is more interesting. The naive randomized algorithm, which flips a coin each time a vertex v $ \in $ V arrives and chooses a random match for v based on this coin flip, does no better than the deterministic algorithms.
However, an equally simple, but subtly different, algorithm does significantly better, achieving a competitive ratio of 1 - 1/e $ \approx $ 0.63. This algorithm, called Ranking, initially chooses a single random ranking  on the vertices in V that is used to choose matches throughout the entire run of the algorithm.\par
Till now, Ranking algorithm is best known algorithm for Online bipartite matching with competitive ratio of (1-1/e).




% You must have at least 2 lines in the paragraph with the drop letter
% (should never be an issue)

% \hfill mds
 
% \hfill August 26, 2015

\section{Problem Statement}
\noindent We can define the online matching problem as follows:\par
Given as input a bipartite graph G = (U, V, E) in which each vertex $ v \in V $ arrives in online fashion, devise an algorithm that matches v to one of its previously unmatched neighbours in U. The matching has to be immediate and is irrevocable, once made. \\
\includegraphics[width=8cm, height=6cm]{Screenshot (39).png}
\\
\textbf{The objective is to maximize the size of the resulting matching so that competitive ratio is maximum(close to 1)}.
Let $|$ V $|$ = $|$ U $|$ = n. We make the assumption that the input graph G has a perfect matching i.e. a matching of size n.
\par
We can define the competitive ratio as follows,
$ Competitive Ratio= min_I ( E[|M|]/n ) $. where ,(instance) I=G(n,n) having a perfect matching and arrival order. The term , E[M] is expected size of matching and n is number of vertices in each set U and V. \\
We will discuss ranking algorithm on example of two sets, Goods(set U) and Buyers(set V). Buyer arrives in online fashion one by one. Edges are revealed once a buyer arrived. Immediately algorithm will match buyer to a available unmatched goods. Now going ahead, we will discuss ranking algorithm, a supporting Lemma and, proof for the competitive ratio of ranking algorithm.




\section{The RANKING Algorithm} 
\noindent The algorithm runs in two steps. In the first step, it chooses a random ranking on vertices in U . In the second step, it matches each vertex upon arrival to an unmatched neighbour deterministically.\\
\par
\noindent The algorithm is as follows:\\
%\section*{\normalfont \textbf{Algorithm: RANKING}}
\noindent \textbf {Initialization}: Pick a random permutation(ranking) 
$ \sigma $ of the vertices in U.\par
\noindent \textbf{Online Matching}:\par
\noindent For each $ v \in V $ that arrives:\par
\noindent Let N(v) be the set of unmatched neighbours of v.\par
\noindent If N(v) $ \neq $ $ \phi $, match u to the vertex u $ \in $ N(v) such that $ \sigma $(u) is minimum.
\\
\par
Lets explore the two steps of the RANKING algorithm.\\
 --- first,randomely permute the goods(once).\\
 --- then, match each arrived buyer to highest available goods.
\\
\includegraphics[width=6.5cm, height=6.5cm]{matrix.jpg}
\includegraphics[width=6.5cm, height=6.5cm]{Screenshot (35).png}
\\

\textbf{Question 1: What happen on permute goods ?} \\
--- After permuting the good once, all previous column also have the same column of red entries.If previous column have the bad matches then this improves chance of good match now and this is the key thing of RANKING algorithm. This is called \textbf{Self Correcting}( i.e. early bad matches improves chance of good matches later).
\\
\par
In 1990 only, Competitive ratio of the RANKING algorithm was proved to be (1-1/e). But Analysis was very difficult at that time.There has been many simplification given over the time.\\
(1)Goel and Mehta, 2008.\\
(2)Birnbaum and Mathieu ,2008\\
(3)Devanur, Jain and Kleinberg ,2013 (Randomize primal-dual)\\
(4)Eden, Feldman,Fiat and Segal 2021 (economic interpretation)\\

\par
\textbf{Question 2: Prove that competitive ratio for RANKING algorithm is (1-1/e)?} \\
---Lets discuss one of the easiet proof which has given by Eden, Feldman,Fiat and Segal in 2021 called economic interpretation.\\
\par
for every good j: pick $w_j$ $\in$ [0,1] at random.\\
then  $p_j$=$e^(w_j -1)$ (price of good j which is rank of the good). \\
---Match each buyer to cheapest available good.\\
---Buyer have unit demand and 0/1 valuations for goods.\\
---if buyer i is matched to good j: gain of 1 in matching.\\
then , revenue $r_j$=$p_j$ and utility $u_i$=1-$p_j$.\\
$u_i$=0 when i is unmatched.\\
$r_j$=0 when j is unmatched.\\
from definition of $p_j$ , $p_j$ $\in$ [1/e,1] and $u_i$ $\in$ [0,1-1/e].\\

\textbf{Lemma 1.(Veri-Central lemma)} Corresponding to each edge e=(i,j):
E[$u_i$ + $r_j$] $>$= 1-1/e (proof of Lemma 1 is in question 3).\\
\\
Now, E[$|$ M $|$]=  \sum_{i=Buyers}^{} E[$u_i$]  +  \sum_{j=Goods}^{} E[$r_j$]\\
                   =  \sum_{(i,j)}^{} E[$u_i$ + $r_j$] \\
                  $>$= n.(1-1/e) 
\\
Hence, Competitive ratio(RANKING) =E[$|$ M $|$] / n \\
                                = n.(1-1/e)/n \\
                                  =(1-1/e).\\
                                  
\textbf{Question 3. Give the proof of Lemma 1?}.\\

---For good j let, $G_j$=G-j;
Then, two runs of algorithm are defined as follows.\\
(1)R: run on j\\
(2)for every good j: 
                 $R_j$: run on $G_j$
\\
\includegraphics[width=6.5cm, height=6.5cm]{Screenshot (37).png}
\\
for every edge e=(i,j), Lets define threshold random variable $u_e$=utility of i in run $R_j$. Therefore, for each edge we have $u_e$.\\
Fact: $u_i$ $>=$ $u_e$ (as i has more options in R than $R_j$ ).\\

\textbf{Lemma 2.} If $p_j$ $<$ 1-$u_e$, then j will definitely get matched in R.\\
---If j is not already matched when i arrives, then j is maximum-utility good for i.\\

Now, as $p_j$ $\in$ [1/e,1]. for 1/e$<=$ $p_j$ $<=$ 1-$u_e$ there is always a matching .Hence there is surely revenue generated for this range.\\
Lets calculate total revenue generated for above range.\\
(1)let z $\in$ [0,1-1/e] condition on $u_e$=z.\\
(2)let w be the value of x at which $p_j$=$e^(x-1)$=1-z.
\\
\includegraphics[width=6.5cm, height=6.5cm]{graph.jpg}.
\\
Area under curve = conditional expectation E[$r_j$ / $u_e$=z] $>$= \int_{x=0}^{x=w} $e^(x-1)$ \,dx =$e^(w-1)$ -1/e=1-z-1/e=1-1/e-z.\\
\par
To compute E[$r_j$] from conditional expectation E[$r_j$ /$u_e$=z] ,\\
E[$r_j$] = E[E[$r_j$ /$u_e$=z]]= \int_{z=0}^{z=1/e} E[$r_j$ / $u_e$=z] .$f_$u_e$(z)$ \,dz =\int_{z=0}^{z=1/e} (1-1/e-z).$f_$u_e$(z)$ \,dz =1-1/e-E[$u_e$].\\
 Now, we have E[$r_j$]=1-1/e-E[$u_e$]\\
  E[$r_j$] + E[$u_e$] =1-1/e \\
  hence, E[$r_j$] + E[$u_i$] $>$=1-1/e    . (as $u_i$ $>$= $u_e$ ,hence E[$u_i$] $>$= E[$u_e$])\\
  Or, E[$r_j$ + $u_i$ ] $>$=1-1/e
  
  














% An example of a floating figure using the graphicx package.
% Note that \label must occur AFTER (or within) \caption.
% For figures, \caption should occur after the \includegraphics.
% Note that IEEEtran v1.7 and later has special internal code that
% is designed to preserve the operation of \label within \caption
% even when the captionsoff option is in effect. However, because
% of issues like this, it may be the safest practice to put all your
% \label just after \caption rather than within \caption{}.
%
% Reminder: the "draftcls" or "draftclsnofoot", not "draft", class
% option should be used if it is desired that the figures are to be
% displayed while in draft mode.
%
%\begin{figure}[!t]
%\centering
%\includegraphics[width=2.5in]{myfigure}
% where an .eps filename suffix will be assumed under latex, 
% and a .pdf suffix will be assumed for pdflatex; or what has been declared
% via \DeclareGraphicsExtensions.
%\caption{Simulation results for the network.}
%\label{fig_sim}
%\end{figure}

% Note that the IEEE typically puts floats only at the top, even when this
% results in a large percentage of a column being occupied by floats.


% An example of a double column floating figure using two subfigures.
% (The subfig.sty package must be loaded for this to work.)
% The subfigure \label commands are set within each subfloat command,
% and the \label for the overall figure must come after \caption.
% \hfil is used as a separator to get equal spacing.
% Watch out that the combined width of all the subfigures on a 
% line do not exceed the text width or a line break will occur.
%
%\begin{figure*}[!t]
%\centering
%\subfloat[Case I]{\includegraphics[width=2.5in]{box}%
%\label{fig_first_case}}
%\hfil
%\subfloat[Case II]{\includegraphics[width=2.5in]{box}%
%\label{fig_second_case}}
%\caption{Simulation results for the network.}
%\label{fig_sim}
%\end{figure*}
%
% Note that often IEEE papers with subfigures do not employ subfigure
% captions (using the optional argument to \subfloat[]), but instead will
% reference/describe all of them (a), (b), etc., within the main caption.
% Be aware that for subfig.sty to generate the (a), (b), etc., subfigure
% labels, the optional argument to \subfloat must be present. If a
% subcaption is not desired, just leave its contents blank,
% e.g., \subfloat[].


% An example of a floating table. Note that, for IEEE style tables, the
% \caption command should come BEFORE the table and, given that table
% captions serve much like titles, are usually capitalized except for words
% such as a, an, and, as, at, but, by, for, in, nor, of, on, or, the, to
% and up, which are usually not capitalized unless they are the first or
% last word of the caption. Table text will default to \footnotesize as
% the IEEE normally uses this smaller font for tables.
% The \label must come after \caption as always.
%
%\begin{table}[!t]
%% increase table row spacing, adjust to taste
%\renewcommand{\arraystretch}{1.3}
% if using array.sty, it might be a good idea to tweak the value of
% \extrarowheight as needed to properly center the text within the cells
%\caption{An Example of a Table}
%\label{table_example}
%\centering
%% Some packages, such as MDW tools, offer better commands for making tables
%% than the plain LaTeX2e tabular which is used here.
%\begin{tabular}{|c||c|}
%\hline
%One & Two\\
%\hline
%Three & Four\\
%\hline
%\end{tabular}
%\end{table}


% Note that the IEEE does not put floats in the very first column
% - or typically anywhere on the first page for that matter. Also,
% in-text middle ("here") positioning is typically not used, but it
% is allowed and encouraged for Computer Society conferences (but
% not Computer Society journals). Most IEEE journals/conferences use
% top floats exclusively. 
% Note that, LaTeX2e, unlike IEEE journals/conferences, places
% footnotes above bottom floats. This can be corrected via the
% \fnbelowfloat command of the stfloats package.




\section{Conclusion}
For the time being, RANKING algorithm is the best algorithm for online bipartite matching with competitive ratio of (1-1/e). Also we can inference that expected matching size in worst case is n(1-1/e) .We can use this ranking algorithm to design algorithms for many open problems which are related to online bipartite matching. For the future, We could refer matching markets discussed in the Vijay V. Vajirani paper to get idea of the open problems in the field of online bipartite matching. 

Open Questions:\\
(1) Is Ranking an optimal algorithm for OBM.? \\
(2) Can we come up with better than RANKING? \\





% conference papers do not normally have an appendix



% use section* for acknowledgment
\ifCLASSOPTIONcompsoc
  % The Computer Society usually uses the plural form
 % \section*{Acknowledgments}
%\else
  % regular IEEE prefers the singular form
% \section*{Acknowledgment}
%\fi


%The authors would like to thank...





% trigger a \newpage just before the given reference
% number - used to balance the columns on the last page
% adjust value as needed - may need to be readjusted if
% the document is modified later
%\IEEEtriggeratref{8}
% The "triggered" command can be changed if desired:
%\IEEEtriggercmd{\enlargethispage{-5in}}

% references section

% can use a bibliography generated by BibTeX as a .bbl file
% BibTeX documentation can be easily obtained at:
% http://mirror.ctan.org/biblio/bibtex/contrib/doc/
% The IEEEtran BibTeX style support page is at:
% http://www.michaelshell.org/tex/ieeetran/bibtex/
%\bibliographystyle{IEEEtran}
% argument is your BibTeX string definitions and bibliography database(s)
%\bibliography{IEEEabrv,../bib/paper}
%
% <OR> manually copy in the resultant .bbl file
% set second argument of \begin to the number of references
% (used to reserve space for the reference number labels box)
\begin{thebibliography}{1}



\bibitem{IEEEhowto:Vazirani}
Vijay V. Vazirani. \emph{ Online Bipartite Matching and Adwords.}, 2022

\bibitem{IEEEhowto:Vazirani}
Richard M. Karp, Umesh V. Vazirani, and Vijay V. Vazirani. \emph{ An optimal algorithm for on-line bipartite matching.} \relax In \emph{STOC}, pages
352–358. ACM, 1990.

\end{thebibliography}




% that's all folks
\end{document}


